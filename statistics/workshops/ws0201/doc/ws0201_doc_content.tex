\documentclass[10pt,letterpaper]{article}

\usepackage[utf8x]{inputenc}
\usepackage{ucs}
\usepackage[spanish]{babel}
\usepackage{amsmath}
\usepackage{amsfonts}
\usepackage{amssymb}
\usepackage{url}
\usepackage{array}
\usepackage[dvipdfm]{graphicx} % Para importar imagenes
\graphicspath{{../img/}} % Ruta de imagenes

\title{Modelos Probabilísticos y Análisis Estadístico\\Modelos de Regresión Lineal}
\author{Mauricio Mejía Castro}

\newbox\bwk\edef\tempd#1pt{#1\string p\string t}\tempd\def\nbextr#1pt{#1}
\def\npts#1{\expandafter\nbextr\the#1\space}
\def\ttwplink#1#2{\special{ps:1 0 0 setrgbcolor}#2\special{ps:0 0 0 setrgbcolor}\setbox\bwk=\hbox{#2}\special{ps:( linkto #1)\space\npts{\wd\bwk} \npts{\dp\bwk} -\npts{\ht\bwk} true\space Cpos}}
\begin{document}
	\maketitle

\section{Descripción del conjunto de datos}
Debido a la naturaleza del problema, se optó por la exclusión de algunas variables que no aportan a la predicción del precio: {\tt id} y {\tt zipcode}. El {\em dataframe} de Los siguientes comandos transforman la variable date a tipo fecha e indican a R que la variable {\tt waterfornt} debe ser considerada tratada categórica:

\begin{verbatim}
	king.test$date <- as.Date(substr(king.test$date, 0, 10))
	king.vali$date <- as.Date(substr(king.vali$date, 0, 10))
	king.test$waterfront <- factor(king.test$waterfront)
	king.vali$waterfront <- factor(king.vali$waterfront)
\end{verbatim}

Con estos elementos podemos plantear un modelo inicial:

\begin{verbatim}
	lmstart <- lm(price ~ ., data = king.test)
	summary(lmstart)
\end{verbatim}

Con lo que obtenemos el siguiente resumen:

\begin{verbatim}
Call:
lm(formula = price ~ ., data = king.test)

Residuals:
Min       1Q   Median       3Q      Max 
-1248356   -99179    -9043    75936  4147237 

Coefficients: (1 not defined because of singularities)
Estimate Std. Error t value Pr(>|t|)    
(Intercept)   -3.858e+07  1.768e+06 -21.827  < 2e-16 ***
date           1.032e+02  1.347e+01   7.662 1.93e-14 ***
bedrooms      -3.183e+04  2.073e+03 -15.358  < 2e-16 ***
bathrooms      3.495e+04  3.598e+03   9.715  < 2e-16 ***
sqft_living    1.486e+02  4.867e+00  30.531  < 2e-16 ***
sqft_lot       1.362e-01  5.332e-02   2.555   0.0106 *  
floors         8.131e+02  3.971e+03   0.205   0.8377    
waterfront1    5.909e+05  2.007e+04  29.437  < 2e-16 ***
view           4.883e+04  2.365e+03  20.645  < 2e-16 ***
condition      3.342e+04  2.613e+03  12.787  < 2e-16 ***
grade          9.727e+04  2.393e+03  40.650  < 2e-16 ***
sqft_above     2.845e+01  4.851e+00   5.864 4.60e-09 ***
sqft_basement         NA         NA      NA       NA    
yr_built      -2.339e+03  8.013e+01 -29.188  < 2e-16 ***
yr_renovated   2.321e+01  4.072e+00   5.700 1.22e-08 ***
lat            5.638e+05  1.163e+04  48.494  < 2e-16 ***
long          -1.143e+05  1.317e+04  -8.679  < 2e-16 ***
sqft_living15  3.144e+01  3.805e+00   8.264  < 2e-16 ***
sqft_lot15    -3.953e-01  8.214e-02  -4.812 1.50e-06 ***
---
Signif. codes:  0 ‘***’ 0.001 ‘**’ 0.01 ‘*’ 0.05 ‘.’ 0.1 ‘ ’ 1

Residual standard error: 199800 on 17271 degrees of freedom
(1 observation deleted due to missingness)
Multiple R-squared:  0.697,	Adjusted R-squared:  0.6967 
F-statistic:  2337 on 17 and 17271 DF,  p-value: < 2.2e-16
\end{verbatim}

Inferimos a partir de la prueba de hipótesis (p-value<2.2e-16) que existe al menos una variable que explica el precio de las casas. Se ve también que la prueba de hipótesis individual para la variable {\tt floors}, cae en la región de rechazo. Por otro lado, la variable {\tt sqft\_basement} no genera resultado alguno. Se excluirán estas dos variables y se recalculará nuevamente el modelo.

\section{Planteamiento del modelo de regresión}

Al ejecutar los siguientes comandos:

\begin{verbatim}
	lmhouses.1 <- lm(price ~ ., data = king.test[, c(-7, -13)])
	summary(lmhouses.1)
\end{verbatim}

Se obtiene el siguiente resultado

\begin{verbatim}
	lm(formula = price ~ ., data = king.test[, c(-7, -13)])
	
	Residuals:
	Min       1Q   Median       3Q      Max 
	-1248494   -99231    -9034    75933  4145675 
	
	Coefficients:
	Estimate Std. Error t value Pr(>|t|)    
	(Intercept)   -3.864e+07  1.741e+06 -22.191  < 2e-16 ***
	date           1.031e+02  1.347e+01   7.660 1.96e-14 ***
	bedrooms      -3.185e+04  2.072e+03 -15.369  < 2e-16 ***
	bathrooms      3.514e+04  3.477e+03  10.106  < 2e-16 ***
	sqft_living    1.483e+02  4.645e+00  31.924  < 2e-16 ***
	sqft_lot       1.359e-01  5.330e-02   2.550   0.0108 *  
	waterfront1    5.910e+05  2.007e+04  29.439  < 2e-16 ***
	view           4.885e+04  2.364e+03  20.669  < 2e-16 ***
	condition      3.338e+04  2.607e+03  12.803  < 2e-16 ***
	grade          9.731e+04  2.385e+03  40.809  < 2e-16 ***
	sqft_above     2.889e+01  4.351e+00   6.638 3.26e-11 ***
	yr_built      -2.335e+03  7.842e+01 -29.778  < 2e-16 ***
	yr_renovated   2.325e+01  4.067e+00   5.717 1.10e-08 ***
	lat            5.640e+05  1.156e+04  48.793  < 2e-16 ***
	long          -1.146e+05  1.304e+04  -8.788  < 2e-16 ***
	sqft_living15  3.134e+01  3.775e+00   8.304  < 2e-16 ***
	sqft_lot15    -3.958e-01  8.209e-02  -4.822 1.44e-06 ***
	---
	Signif. codes:  0 ‘***’ 0.001 ‘**’ 0.01 ‘*’ 0.05 ‘.’ 0.1 ‘ ’ 1
	
	Residual standard error: 199800 on 17272 degrees of freedom
	(1 observation deleted due to missingness)
	Multiple R-squared:  0.697,	Adjusted R-squared:  0.6968 
	F-statistic:  2484 on 16 and 17272 DF,  p-value: < 2.2e-16
\end{verbatim}

A pesar del bajo coeficiente de determinación, el modelo parece exhibir ciertas propiedades deseables. Sin embargo, al graficar los valores predichos contra los residuales, no veremos un comportamiento de homocedasticidad.

\begin{figure}[h!]
	%\includegraphics[scale=0.5]{resvspred1.jpeg}
	\caption{Heterocedasticiad del modelo price \~ .}
\end{figure}

Para solucionar esto, utilizaremos la transformación Box-Cox. Con los siguientes comandos, hallaremos la grafica y calcularemos el lambda máximo (-0.02) al que elevaremos la variable predictora:

\begin{verbatim}
	trhouses <- boxcox(price ~ ., data = king.train[, c(-1, -7, -8, -13)])
	max.lambda <- trhouses$x[which.max(trhouses$y)]
	lmhouses.tr <- lm(price^max.lambda ~ ., 
					  data = king.train[, c(-1, -7, -8, -13)])
	summary(lmhouses.tr)
\end{verbatim}

\begin{figure}[h!]
	%\includegraphics[scale=0.5]{boxcox.jpeg}
	\caption{Gráfico Box Cox}
\end{figure}

Esto nos arroja el siguiente resultado

\begin{verbatim}
	Call:
	lm(formula = price^max.lambda ~ ., data = king.train[, c(-1, 
	-7, -8, -13)])
	
	Residuals:
	Min         1Q     Median         3Q        Max 
	-0.0186987 -0.0025094 -0.0000394  0.0025309  0.0219730 
	
	Coefficients:
	Estimate Std. Error t value Pr(>|t|)    
	(Intercept)    1.882e+00  3.430e-02  54.884  < 2e-16 ***
	bedrooms       1.820e-04  4.112e-05   4.425 9.72e-06 ***
	bathrooms     -1.280e-03  6.910e-05 -18.520  < 2e-16 ***
	sqft_living   -1.937e-06  9.231e-08 -20.988  < 2e-16 ***
	sqft_lot      -8.165e-09  1.059e-09  -7.710 1.33e-14 ***
	view          -1.144e-03  4.386e-05 -26.072  < 2e-16 ***
	condition     -1.029e-03  5.174e-05 -19.881  < 2e-16 ***
	grade         -2.506e-03  4.737e-05 -52.896  < 2e-16 ***
	sqft_above    -3.880e-07  8.643e-08  -4.489 7.19e-06 ***
	yr_built       4.471e-05  1.558e-06  28.690  < 2e-16 ***
	yr_renovated  -7.185e-07  8.067e-08  -8.906  < 2e-16 ***
	lat           -2.150e-02  2.296e-04 -93.606  < 2e-16 ***
	long           1.204e-03  2.592e-04   4.644 3.45e-06 ***
	sqft_living15 -1.478e-06  7.498e-08 -19.706  < 2e-16 ***
	sqft_lot15     6.279e-09  1.631e-09   3.849 0.000119 ***
	---
	Signif. codes:  0 ‘***’ 0.001 ‘**’ 0.01 ‘*’ 0.05 ‘.’ 0.1 ‘ ’ 1
	
	Residual standard error: 0.00397 on 17274 degrees of freedom
	(1 observation deleted due to missingness)
	Multiple R-squared:  0.763,	Adjusted R-squared:  0.7628 
	F-statistic:  3973 on 14 and 17274 DF,  p-value: < 2.2e-16
\end{verbatim}

En este punto, no solo vemos un importante descenso en el Error Residual Estándar, también vemos una leve mejora en el $R^2$ y el $R^2$ ajustado. Veremos también que queda corregida la heterocedasticidad del modelo anterior.

\begin{figure}[h!]
	%\includegraphics[scale=0.5]{residafter.jpeg}
	\caption{Valores residuales contra valores predichos después de la transformación de Box Cox}
\end{figure}

\section{Diagnóstico del modelo}
A continuación validaremos los supuestos que el modelo debe cumplir. Veremos los valores atípicos, los valores con alto {\em leverage} y los valores influyentes. Además, nos aseguraremos que la estructura del modelo se atenga a un comportamiento lineal.

\subsection{Normalidad de los valores residuales}
Podemos evaluar visualmente la normalidad de los valores residuales a través del Q-Q Plot. Este gráfico valida el primer supuesto acerca de la normalidad de los valores residuales.

\begin{figure}[h!]
	%\includegraphics[scale=0.5]{qqplot.jpeg}
	\caption{Gráfico Q-Q Plot}
\end{figure}


\subsection{Valores con alto {\em leverage}}

Para identificar los valores con alto {\em leverage} de manera visual, utilizamos el siguiente comando:

\begin{verbatim}
	infhouses <- influence(lmhouses.tr)
	halfnorm(infhouses$hat, labs = row.names(lmhouses.tr))
\end{verbatim}

\begin{figure}[h!]
	%\includegraphics[scale=0.5]{halfnormal.jpeg}
	\caption{Gráfico para la distancia de Cook}
\end{figure}



\subsection{Valores influyentes}
Con la distancia de Cook, podemos identificar también que en el modelo existen mútliples valores influyentes:

\begin{figure}[h]
	%\includegraphics[scale=0.5]{cook.jpeg}
	\caption{Gráfico para la distancia de Cook}
\end{figure}

Finalmente, la capacidad predictiva del modelo a través de la Raíz del Cuadrado Medio del Error RMSE = 664255.

\end{document}
