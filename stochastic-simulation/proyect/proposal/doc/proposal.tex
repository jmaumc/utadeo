\documentclass[11pt,letterpaper]{scrartcl}
\usepackage[utf8x]{inputenc}
\usepackage[T1]{fontenc}
\usepackage{ucs}
\usepackage[spanish]{babel}
\usepackage{amsmath}
\usepackage{amsfonts}
\usepackage{amssymb}
\usepackage{graphicx}
%%% Maketitle metadata
\newcommand{\horrule}[1]{\rule{\linewidth}{#1}} % Horizontal rule

\title{
	%\vspace{-1in} 	
	\usefont{OT1}{bch}{b}{n}
	\normalfont \normalsize \textsc{Universidad de Bogotá Jorge Tadeo Lozano} \\ [25pt]
	\horrule{0.5pt} \\[0.4cm]
	\huge Simulación Estocástica \\ Propuesta Proyecto de Curso \\
	\horrule{2pt} \\[0.5cm]
}
\author{Mauricio Mejía Castro}
\begin{document}
	\maketitle
	\tableofcontents
	
	\section{Descripción del problema}
	Hola \cite{ross2014introduction}
		\subsection{Contexto}
		En un juego de \textit{squash} participan dos jugadores: el jugador 1 y el jugador 2. El juego consiste en una secuencia de puntos. Si el jugador $i$ sirve y gana el punto, entonces su puntaje aumenta en 1 y retiene el servicio (para $i=$1 o 2). Si el jugador sirve y pierde el punto, entonces el servicio se transfiere al otro jugador y su puntaje permanece igual.
		
		El ganador es la primera persona en alcanzar 9, a menos que todos alcanzen 8 puntos primero. Cuando todos los jugadores alcanzan 8 puntos, el juego continua hasta que alguno logre estar dos puntos por delante. En ese caso, este jugador es el ganador.
		Se propone simular un juego de squash y estimar la probabiblidad de que el jugador 1 gane. Este proyecto es tomado del Capitulo 12 en \cite{jones2009introduction}.
		
		\subsection{Estado del arte}
		Varios autories han aplicado metdodos de simulacion estocastica para estimar la probabilidad de ganar en ciertos deportes. Por ejemplo, una investigacion similar a la del proyecto aca propuesto y llevada a cabo por McGarry y Granks \cite{mcgarry1994stochastic}, trata de predecir el desempeno de los jugadores de squash basado en el analisis de juegos anteriores.
		
		Min et al. \cite{min2008compound} proponene un framework para la prediccion de resultados en futbol a traves de inferencia Bayesiana y razonamiento basado en reglas. Tambien utilizan una aproximacion basada en series de tiempo con conocimieto obtenido del juego. Como resultado los autores afirman obtener predicciones razonables y estables.
		
		En un trabajo de Weninger y Lames \cite{wenninger2016performance} se propone la estimacion de la probabilidad de ganar en  el tenis. Con ello se buscaba disenar estrategias de juego y tacticas que perminitieran mejorar el desempeno de los jugadores. Como resultado conlueyn que los errores y los partidos largos tienen gran impacto en la probabilidad de perder el partido.
		
	\section{Objetivo general}
	Simular un juego de \textit{squash} y estimar la probabilidad de que el jugador 1 gane.
	\section{Metodología}
	f
	Se define:
	\begin{gather}
		a = \mathbb{P}(\text{jugador 1 gana un punto}\,|\,\text{jugador 1 sirve})\\
		b = \mathbb{P}(\text{jugador 1 gana un punto}\,|\,\text{jugador 2 sirve})\\
		x = \text{puntaje del jugador 1}\\
		y = \text{puntaje del jugador 2}\\
		z = \begin{cases}
			1 & \text{si el jugador 1 tiene el servicio}\\
			2 & \text{si el jugador 2 tiene el servicio\text{.}}
		\end{cases} 
	\end{gather}
		\subsection{Planteamiento del modelo}
		\subsection{Diseño experimental}
	\bibliographystyle{plain}
	\bibliography{proposal}
	
	
\end{document}