\documentclass{article}
\usepackage{array}
\renewcommand{\arraystretch}{1.1}
\usepackage{tikz}
\usetikzlibrary{shapes.multipart}
\usetikzlibrary{positioning}
\usetikzlibrary{shadows}
\usetikzlibrary{calc}

\usepackage{pdflscape}

% code for "one to omany", etc. is taken from https://tex.stackexchange.com/q/141797/101651
\makeatletter
\pgfarrowsdeclare{crow's foot}{crow's foot}
{
    \pgfarrowsleftextend{+-.5\pgflinewidth}%
    \pgfarrowsrightextend{+.5\pgflinewidth}%
}
{
    \pgfutil@tempdima=0.6pt%
    \pgfsetdash{}{+0pt}%
    \pgfsetmiterjoin%
    \pgfpathmoveto{\pgfqpoint{0pt}{-9\pgfutil@tempdima}}%
    \pgfpathlineto{\pgfqpoint{-13\pgfutil@tempdima}{0pt}}%
    \pgfpathlineto{\pgfqpoint{0pt}{9\pgfutil@tempdima}}%
    \pgfpathmoveto{\pgfqpoint{0\pgfutil@tempdima}{0\pgfutil@tempdima}}%
    \pgfpathmoveto{\pgfqpoint{-8pt}{-6pt}}% 
    \pgfpathlineto{\pgfqpoint{-8pt}{-6pt}}%  
    \pgfpathlineto{\pgfqpoint{-8pt}{6pt}}% 
    \pgfusepathqstroke%
}

\pgfarrowsdeclare{omany}{omany}
{
    \pgfarrowsleftextend{+-.5\pgflinewidth}%
    \pgfarrowsrightextend{+.5\pgflinewidth}%
}
{
    \pgfutil@tempdima=0.6pt%
    \pgfsetdash{}{+0pt}%
    \pgfsetmiterjoin%
    \pgfpathmoveto{\pgfqpoint{0pt}{-9\pgfutil@tempdima}}%
    \pgfpathlineto{\pgfqpoint{-13\pgfutil@tempdima}{0pt}}%
    \pgfpathlineto{\pgfqpoint{0pt}{9\pgfutil@tempdima}}%
    \pgfpathmoveto{\pgfqpoint{0\pgfutil@tempdima}{0\pgfutil@tempdima}}%  
    \pgfpathmoveto{\pgfqpoint{0\pgfutil@tempdima}{0\pgfutil@tempdima}}%
    \pgfpathmoveto{\pgfqpoint{-6pt}{-6pt}}% 
    \pgfpathcircle{\pgfpoint{-11.5pt}{0}} {3.5pt}
    \pgfusepathqstroke%
}

\pgfarrowsdeclare{one}{one}
{
    \pgfarrowsleftextend{+-.5\pgflinewidth}%
    \pgfarrowsrightextend{+.5\pgflinewidth}%
}
{
    \pgfutil@tempdima=0.6pt%
    \pgfsetdash{}{+0pt}%
    \pgfsetmiterjoin%
    \pgfpathmoveto{\pgfqpoint{0\pgfutil@tempdima}{0\pgfutil@tempdima}}%
    \pgfpathmoveto{\pgfqpoint{-6pt}{-6pt}}% 
    \pgfpathlineto{\pgfqpoint{-6pt}{-6pt}}%  
    \pgfpathlineto{\pgfqpoint{-6pt}{6pt}}% 
    \pgfpathmoveto{\pgfqpoint{0\pgfutil@tempdima}{0\pgfutil@tempdima}}%
    \pgfpathmoveto{\pgfqpoint{-8pt}{-6pt}}% 
    \pgfpathlineto{\pgfqpoint{-8pt}{-6pt}}%  
    \pgfpathlineto{\pgfqpoint{-8pt}{6pt}}%    
    \pgfusepathqstroke%
}
\makeatother

\tikzset{%
    pics/entity/.style n args={3}{code={%
        \node[draw,
        rectangle split,
        rectangle split parts=2,
        text height=1.5ex,
        ] (#1)
        {#2 \nodepart{second}
            \begin{tabular}{>{\raggedright\arraybackslash}p{8.5em}}
                #3
            \end{tabular}
        };%
    }},
    pics/entitynoatt/.style n args={2}{code={%
        \node[draw,
        text height=1.5ex,
        ] (#1)
        {#2};%
    }},
    zig zag to/.style={
        to path={(\tikztostart) -| ($(\tikztostart)!#1!(\tikztotarget)$) |- (\tikztotarget)}
    },
    zig zag to/.default=0.5,   
    one to one/.style={
        one-one, zig zag to
    },
    one to oone/.style={            % I do not how to make "one to Optional-one" rel
        one-one, zig zag to
    },
    one to many/.style={
        one-crow's foot, zig zag to,
    },
    one to omany/.style={
        one-omany, zig zag to
    }
}

\begin{document}

\begin{landscape}
    \begin{center}
        \begin{tikzpicture}
            \pic {entity={A}{Entity A}{%
                attribute 1 \\
                attribute 2 \\        
                ... \\
                attribute i
            }};
            \pic[right=7em of A] {entity={AB}{Entity A for Entity B}{%
                attribute 1 \\
                attribute 2
            }};
            \pic[right=7em of AB] {entity={B}{Entity B}{%
                attribute 1 \\
                attribute 2 \\        
                ... \\
                attribute i
            }};
            \pic[below=16ex of B] {entity={C}{Entity C}{%
                attribute 1    
            }};
            \pic[below=15ex of AB] {entitynoatt={C1}{Entity C\_1}};
            \pic[below=9ex of C1] {entitynoatt={C2}{Entity C\_2}};
            \draw[one to omany] (A.east) -- node[above]{\footnotesize is in} (AB.west);
            \draw[one to omany] (B.west) -- node[above]{\footnotesize is in} (AB.east);
            \draw[one to one] (B.south) -- node[right]{\footnotesize is in} (C.north);
            \draw[one to oone] (C.west) -| ($(C.west)!.5!(C1.east)$) |- node[above]{\footnotesize is in} (C1.east);  % Make "oone" rel
            \draw[one to oone] (C.west) -| ($(C.west)!.5!(C2.east)$) |- node[below]{\footnotesize is in} (C2.east);  % Make "oone" rel
            \end{tikzpicture}
    \end{center}
\end{landscape}

\end{document}